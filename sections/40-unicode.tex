% -----------------------------------------------
% chktex-file 44
\documentclass[../index.tex]{subfiles}

% -----------------------------------------------

\begin{document}

% -----------------------------------------------
\renewcommand{\sectiontitle}{Unicode}
\section{\sectiontitle}
% I hope you found ASCII pretty simple, because it's only going to get more complicated from
% here on out.
%
% Why's that?
% Because language is complicated.

% ---------------------------
\renewcommand{\currenttitle}{The problems with ASCII}
\begin{frame}{\currenttitle}
% Let's review the problems with ASCII again.
% Knowing these shortcomings will help us understand how and why Unicode is designed the way
% it is.
%
% The original ASCII character set only supported the basic English alphabet, the ten Arabic
% numeral digits, some punctuations symbols, and some other unreadable symbols.
% As we've mentioned before, this isn't really going to cut it once you move outside of English.
%
% The extended ASCII character sets like Latin 1 tried to help this, but then we have many
% different widely adopted but largely incompatible character sets.
% When you're transforming data across the globe into different systems using different
% character sets and encodings, this is going to be a problem.
%
% And as we'll see in a brief examination of other languages, the terminology and design
% behind ASCII is too naïve to support many kinds of languages.
  Let's review: \\

  \begin{itemize}
    \item[--] Minimal language support
    \item[--] Variants that are widespread but incompatible
    \item[--] Too naïve for many types of languages
  \end{itemize}
\end{frame}

% ---------------------------
\renewcommand{\currenttitle}{Even more precise terminology}
\begin{frame}{\currenttitle}
\end{frame}

% ---------------------------
\renewcommand{\currenttitle}{How does Unicode define `character'?}
\begin{frame}{\currenttitle}
  ``The smallest component of written language that has semantic value; refers to the
    abstract meaning and/or shape, rather than a specific shape\ldots''
\end{frame}

% ---------------------------
\renewcommand{\currenttitle}{The Universal Character Set}
\begin{frame}{\currenttitle}
\end{frame}

% ---------------------------
\renewcommand{\currenttitle}{The seventeen Unicode planes}
\begin{frame}{\currenttitle}
\end{frame}

% ---------------------------
\renewcommand{\currenttitle}{Examining UTF-8 encoding}
\begin{frame}{\currenttitle}
\end{frame}

% ---------------------------
\renewcommand{\currenttitle}{Writing a UTF-8 decoder}
\begin{frame}{\currenttitle}
\end{frame}

% ---------------------------
\renewcommand{\currenttitle}{UTF-16}
\begin{frame}{\currenttitle}
\end{frame}

% ---------------------------
\renewcommand{\currenttitle}{Unicode's shortcomings}
\begin{frame}{\currenttitle}
\end{frame}

% -----------------------------------------------

\end{document}
