% -----------------------------------------------
\documentclass[../index.tex]{subfiles}

% -----------------------------------------------

\begin{document}

% -----------------------------------------------
\renewcommand{\sectiontitle}{ASCII}
\section{\sectiontitle}
% There were other text encodings before ASCII, but ASCII is one that is still used sometimes.
% Today, the international standard of text encoding is Unicode.
%
% We'll examine the ASCII standard and its limitations, and then look at how Unicode attempts
% to ameliorate these issues.

% ---------------------------
\renewcommand{\currenttitle}{1968 \textendash{} ASCII}
\begin{frame}{\currenttitle}
% The text encoding standard ASCII was published in 1968. It was based on telegraph code.
% Designed for use of English, it encodes 0 - 9, a - z, A - Z, and punctuation symbols like
% the period and parenthesis. It also has characters such as the line feed and carriage return.
%
% The designers of ASCII, like with text encodings that came before it, assigned each character
% a number.
% So when we store text according to the ASCII encoding, we convert each character into its
% corresponding numerical value, and then store that value in binary.
  \textbf{ASCII} \textendash{} American Standard Code for Information Interchange \\
  \vspace*{1em}
  Each character is assigned a single value.
\end{frame}

% ---------------------------
\begin{frame}{\currenttitle}
% See here a table of all the characters encoded in the ASCII standard. This is first page.
%
% In the first column, we see the decimal value assigned to each character.
% In the second column, we see the hex equivalent.
%
% The lowest values are some non-printable characters.
%
% Notice especially the NUL character, which was assigned the value of zero.
% The NUL character is pretty important in a a variety of different domains, such as printing.
% In the C programming language and some data formats, the NUL character indicates the
% termination of a string.
%
% Notice the numerical characters 0 to 9 are assigned values from 48 to 57.
  \vspace*{1em}
  \scriptsize
  \begin{figure}
    \begin{table}
      \begin{tabular}{|c|c|c||c|c|c||c|c|c||c|c|c|}                       \hline
        Dec & Hex &    & Dec & Hex &    & Dec & Hex &   & Dec & Hex &  \\ \hline
        0  & 00 & NUL  & 16 & 10 & DLE  & 32 & 20 &     & 48 & 30 & 0  \\
        1  & 01 & SOH  & 17 & 11 & DC1  & 33 & 21 & !   & 49 & 31 & 1  \\
        2  & 02 & STX  & 18 & 12 & DC2  & 34 & 22 & "   & 50 & 32 & 2  \\
        3  & 03 & ETX  & 19 & 13 & DC3  & 35 & 23 & #   & 51 & 33 & 3  \\
        4  & 04 & EOT  & 20 & 14 & DC4  & 36 & 24 & \$  & 52 & 34 & 4  \\
        5  & 05 & ENQ  & 21 & 15 & NAK  & 37 & 25 & \%  & 53 & 35 & 5  \\
        6  & 06 & ACK  & 22 & 16 & SYN  & 38 & 26 & \&  & 54 & 36 & 6  \\
        7  & 07 & BEL  & 23 & 17 & ETB  & 39 & 27 & '   & 55 & 37 & 7  \\
        8  & 08 & BS   & 24 & 18 & CAN  & 40 & 28 & (   & 56 & 38 & 8  \\
        9  & 09 & HT   & 25 & 19 & EM   & 41 & 29 & )   & 57 & 39 & 9  \\
        10 & 0A & LF   & 26 & 1A & SUB  & 42 & 2A & *   & 58 & 3A & :  \\
        11 & 0B & VT   & 27 & 1B & ESC  & 43 & 2B & +   & 59 & 3B & ;  \\
        12 & 0C & FF   & 28 & 1C & FS   & 44 & 2C & ,   & 60 & 3C & <  \\
        13 & 0D & CR   & 29 & 1D & GS   & 45 & 2D & -   & 61 & 3D & =  \\
        14 & 0E & SO   & 30 & 1E & RS   & 46 & 2E & .   & 62 & 3E & >  \\
        15 & 0F & SI   & 31 & 1F & US   & 47 & 2F & /   & 63 & 3F & ?  \\ \hline
      \end{tabular}
    \end{table}
    \caption{ASCII table, produced with \texttt{ascii | tail -17}}
  \end{figure}
  \normal
\end{frame}

% ---------------------------
\begin{frame}{\currenttitle}
% These are the rest of the characters specified by the ASCII standard.
%
% Notice that the alphabeta starts on 65 with uppercase characters.
% Then there are a few symbols, and then the lowercase characters from 97 to 122.
  \vspace*{1em}
  \scriptsize
  \begin{figure}
    \begin{table}
      \begin{tabular}{|c|c|c||c|c|c||c|c|c||c|c|c|}                                                 \hline
        Dec & Hex &  & Dec & Hex &                & Dec & Hex  &  & Dec & Hex  &                 \\ \hline
        64 & 40 & @  & 80 & 50 & P                & 96 & 60  & `  & 112 & 70 & p                 \\
        65 & 41 & A  & 81 & 51 & Q                & 97 & 61  & a  & 113 & 71 & q                 \\
        66 & 42 & B  & 82 & 52 & R                & 98 & 62  & b  & 114 & 72 & r                 \\
        67 & 43 & C  & 83 & 53 & S                & 99 & 63  & c  & 115 & 73 & s                 \\
        68 & 44 & D  & 84 & 54 & T                & 100 & 64 & d  & 116 & 74 & t                 \\
        69 & 45 & E  & 85 & 55 & U                & 101 & 65 & e  & 117 & 75 & u                 \\
        70 & 46 & F  & 86 & 56 & V                & 102 & 66 & f  & 118 & 76 & v                 \\
        71 & 47 & G  & 87 & 57 & W                & 103 & 67 & g  & 119 & 77 & w                 \\
        72 & 48 & H  & 88 & 58 & X                & 104 & 68 & h  & 120 & 78 & x                 \\
        73 & 49 & I  & 89 & 59 & Y                & 105 & 69 & i  & 121 & 79 & y                 \\
        74 & 4A & J  & 90 & 5A & Z                & 106 & 6A & j  & 122 & 7A & z                 \\
        75 & 4B & K  & 91 & 5B & \lbrack{}        & 107 & 6B & k  & 123 & 7B & \{                \\
        76 & 4C & L  & 92 & 5C & \textbackslash{} & 108 & 6C & l  & 124 & 7C & |                 \\
        77 & 4D & M  & 93 & 5D & \rbrack{}        & 109 & 6D & m  & 125 & 7D & \}                \\
        78 & 4E & N  & 94 & 5E & \^{}             & 110 & 6E & n  & 126 & 7E & \textasciitilde{} \\
        79 & 4F & O  & 95 & 5F & \_               & 111 & 6F & o  & 127 & 7F & DEL               \\ \hline
      \end{tabular}
    \end{table}
    \caption{ASCII table, cont.}
  \end{figure}
  \normal
\end{frame}

% ---------------------------
\begin{frame}{\currenttitle}
% There are 16 rows, and 8 column groups. That means there are 128 characters encoded.
\end{frame}

% ---------------------------
\begin{frame}{How does Unicode define a character?}
  \only{}
\end{frame}


% -----------------------------------------------

\end{document}
