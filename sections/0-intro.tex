% -----------------------------------------------
\documentclass[../index.tex]{subfiles}

\usepackage{tikz}
\usetikzlibrary{arrows,shapes.gates.logic.US,shapes.gates.logic.IEC,calc}
% -----------------------------------------------

\begin{document}
\newcommand{\indentt}{\hspace*{2em}}
\newcommand{\sectiontitle}{}
\newcommand{\currenttitle}{}

% -----------------------------------------------
\renewcommand{\sectiontitle}{It's just text, right?}
\section{\sectiontitle}

% ---------------------------
\begin{frame}{\sectiontitle}
  \only<+>{What's a string?}
  \only<+>{What's a character?}
\end{frame}


% -----------------------------------------------
\renewcommand{\sectiontitle}{A little bit of linguistics and typography}
\section{\sectiontitle}

% ---------------------------
\begin{frame}{\sectiontitle}
  \only<+->{A \textbf{character} is the basic symbol used to write or print a language} \\
  \only<+->{A \textbf{character set} is a collection of characters to write a language} \\
  \only<+->{A \textbf{glyph} is the \textit{visual} representation of a character}
\end{frame}


% -----------------------------------------------
\renewcommand{\sectiontitle}{A memory refresher}
\section{\sectiontitle}
% Before we get into how computers handle text, we need to talk about memory.

% ---------------------------
\renewcommand{\currenttitle}{Bits and bytes}
\begin{frame}{\currenttitle}
% Computers store information bits, a portmanteau of "binary digit", which represent an "on"
% and an "off" state in the abstract sense.
%
% In the physical device, this might be two stable states of a flip-flop, two positions of an
% electrical switch, two distinct voltage or current levels allowed by a circuit, two distinct
% levels of light intensity.
  \only<+->{Computers store information in \textbf{bits} (0 or 1 / \texttt{TRUE} or \texttt{FALSE})} \\
\end{frame}

% ---------------------------
\begin{frame}{\currenttitle}
% We can do binary logic operations on bits, which are the basis for other operations such as
% addition and subtraction.
  \newcommand{\false}{\texttt{0}}
  \newcommand{\true}{\texttt{1}}
  \newcommand{\m}[1]{\texttt{#1}}

  We can do binary logic and arithmetic with these:

  \begin{table}
    \begin{tabular}{c c c c}
      \m{a} & \m{b} & op & result \\
      \hline{}
      \false{} & \true{} & \m{AND} & \false{} \\
      \false{} & \true{} & \m{OR} & \true{} \\
      \true{} & \true{} & \m{NAND} & \false{} \\
      \false{} & \false{} & \m{NOR} & \false{} \\
      \true{} & \true{} & \m{XOR} & \false{}
    \end{tabular}
    \caption{Logic binop inputs and results}
  \end{table}
\end{frame}

% ---------------------------
% We often see these operations visualized as logic gates.
\begin{frame}{\currenttitle}
  We often see these operations visualized as logic gates:
  \begin{figure}
    \begin{circuitikz} \draw
      (0,0) node[xor port] (xor) {}
      (2,0) node[and port] (and) {};
    \end{circuitikz}
    \caption{XOR and AND logic gates}
  \end{figure}
\end{frame}

% ---------------------------
\begin{frame}{\currenttitle}
  \only<+->{A \textbf{byte} is equal to 8 bits} \\
\end{frame}


% -----------------------------------------------
\renewcommand{\sectiontitle}{ASCII \textrightarrow{} Unicode}
\section{\sectiontitle}
% There were other text encodings before ASCII, but ASCII is one that is still used sometimes.
% Today, the international standard of text encoding is Unicode.
%
% We'll examine the ASCII standard and its limitations, and then look at how Unicode attempts
% to ameliorate these issues.

% ---------------------------
\begin{frame}{1968 \textendash{} ASCII}
  \only{}
\end{frame}

\begin{frame}{How does Unicode define a character?}
  \only{}
\end{frame}


% -----------------------------------------------

\end{document}
